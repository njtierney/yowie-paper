\documentclass[12pt,a4paper,]{article}
\usepackage{float}
%\usepackage[sfdefault,lf,t]{carlito}
\usepackage[default,tabular,lf]{sourcesanspro}
\usepackage[cmintegrals]{newtxsf}
\usepackage[italic,eulergreek]{mathastext}
\usepackage{setspace}

\usepackage{amssymb,amsmath}
\usepackage{ifxetex,ifluatex}
\usepackage{fixltx2e} % provides \textsubscript
\ifnum 0\ifxetex 1\fi\ifluatex 1\fi=0 % if pdftex
  \usepackage[T1]{fontenc}
  \usepackage[utf8]{inputenc}
  \usepackage{csquotes}
\else % if luatex or xelatex
  \usepackage{unicode-math}
  \defaultfontfeatures{Ligatures=TeX,Scale=MatchLowercase}
\fi
% use upquote if available, for straight quotes in verbatim environments
\IfFileExists{upquote.sty}{\usepackage{upquote}}{}
% use microtype if available
\IfFileExists{microtype.sty}{%
\usepackage[]{microtype}
\UseMicrotypeSet[protrusion]{basicmath} % disable protrusion for tt fonts
}{}
\PassOptionsToPackage{hyphens}{url} % url is loaded by hyperref
\usepackage{fancybox}
\usepackage[unicode=true]{hyperref}
\hypersetup{
            pdftitle={Response to reviewers},
            pdfborder={0 0 0},
            breaklinks=true}
\urlstyle{same}  % don't use monospace font for urls
\usepackage{geometry}
\geometry{left=2cm,right=2cm,top=2.5cm,bottom=2.5cm}
\usepackage[style=authoryear-comp,backend=biber, natbib=true,]{biblatex}
\usepackage{longtable,booktabs}
% Fix footnotes in tables (requires footnote package)
\IfFileExists{footnote.sty}{\usepackage{footnote}\makesavenoteenv{long table}}{}
\IfFileExists{parskip.sty}{%
\usepackage{parskip}
}{% else
\setlength{\parindent}{0pt}
\setlength{\parskip}{6pt plus 2pt minus 1pt}
}
\setlength{\emergencystretch}{3em}  % prevent overfull lines
\providecommand{\tightlist}{%
  \setlength{\itemsep}{0pt}\setlength{\parskip}{0pt}}
\setcounter{secnumdepth}{5}

% set default figure placement to htbp
\makeatletter
\def\fps@figure{htbp}
\makeatother

\title{Response to reviewers}


%% MONASH STUFF

%% CAPTIONS
\RequirePackage{caption}
\DeclareCaptionStyle{italic}[justification=centering]
 {labelfont={bf},textfont={it},labelsep=colon}
\captionsetup[figure]{style=italic,format=hang,singlelinecheck=true}
\captionsetup[table]{style=italic,format=hang,singlelinecheck=true}

%% FONT
\usepackage{bm,url}

%% HEADERS AND FOOTERS
\RequirePackage{fancyhdr}
\pagestyle{fancy}
\lfoot{}\cfoot{}\rfoot{}
\lhead{\textsf{Response to reviewers}}
\rhead{\textsf{\thepage}}
\setlength{\headheight}{15pt}
\renewcommand{\headrulewidth}{0.4pt}
\fancypagestyle{plain}{%
\fancyhf{} % clear all header and footer fields
\fancyfoot[C]{\sffamily\thepage} % except the center
\renewcommand{\headrulewidth}{0pt}
\renewcommand{\footrulewidth}{0pt}}

%% MATHS
\RequirePackage{bm,amsmath}
\allowdisplaybreaks

%% GRAPHICS
\RequirePackage{graphicx}
\setcounter{topnumber}{2}
\setcounter{bottomnumber}{2}
\setcounter{totalnumber}{4}
\renewcommand{\topfraction}{0.85}
\renewcommand{\bottomfraction}{0.85}
\renewcommand{\textfraction}{0.15}
\renewcommand{\floatpagefraction}{0.8}

%\RequirePackage[section]{placeins}

%% SECTION TITLES
\RequirePackage[compact,sf,bf]{titlesec}
\titleformat{\section}[block]
  {\fontsize{15}{17}\bfseries\sffamily}
  {\thesection}
  {0.4em}{}
\titleformat{\subsection}[block]
  {\fontsize{12}{14}\bfseries\sffamily}
  {\thesubsection}
  {0.4em}{}
\titlespacing{\section}{0pt}{*3}{*1}
\titlespacing{\subsection}{0pt}{*1}{*0.5}

%% LINE AND PAGE BREAKING
\sloppy
\raggedbottom
\usepackage[bottom]{footmisc}
\clubpenalty = 10000
\widowpenalty = 10000
\brokenpenalty = 10000
\RequirePackage{microtype}

%% HYPERLINKS
\RequirePackage{xcolor} % Needed for links
\definecolor{darkblue}{rgb}{0,0,.6}
\RequirePackage{url}

\makeatletter
\@ifpackageloaded{hyperref}{}{\RequirePackage{hyperref}}
\makeatother
\hypersetup{
     citecolor=0 0 0,
     breaklinks=true,
     bookmarksopen=true,
     bookmarksnumbered=true,
     linkcolor=darkblue,
     urlcolor=blue,
     citecolor=darkblue,
     colorlinks=true}

\usepackage[showonlyrefs]{mathtools}

%% BIBLIOGRAPHY

\makeatletter
\@ifpackageloaded{biblatex}{}{\usepackage[style=authoryear-comp, backend=biber, natbib=true]{biblatex}}
\makeatother
\ExecuteBibliographyOptions{bibencoding=utf8,minnames=1,maxnames=3, maxbibnames=99,dashed=false,terseinits=true,giveninits=true,uniquename=false,uniquelist=false,doi=false, isbn=false,url=true,sortcites=false}
\DeclareFieldFormat{url}{\texttt{\url{#1}}}
\DeclareFieldFormat[article]{pages}{#1}
\DeclareFieldFormat[inproceedings]{pages}{\lowercase{pp.}#1}
\DeclareFieldFormat[incollection]{pages}{\lowercase{pp.}#1}
\DeclareFieldFormat[article]{volume}{\mkbibbold{#1}}
\DeclareFieldFormat[article]{number}{\mkbibparens{#1}}
\DeclareFieldFormat[article]{title}{\MakeCapital{#1}}
\DeclareFieldFormat[article]{url}{}
%\DeclareFieldFormat[book]{url}{}
%\DeclareFieldFormat[inbook]{url}{}
%\DeclareFieldFormat[incollection]{url}{}
%\DeclareFieldFormat[inproceedings]{url}{}
\DeclareFieldFormat[inproceedings]{title}{#1}
\DeclareFieldFormat{shorthandwidth}{#1}
%\DeclareFieldFormat{extrayear}{}
% No dot before number of articles
\usepackage{xpatch}
\xpatchbibmacro{volume+number+eid}{\setunit*{\adddot}}{}{}{}
% Remove In: for an article.
\renewbibmacro{in:}{%
  \ifentrytype{article}{}{%
  \printtext{\bibstring{in}\intitlepunct}}}
\AtEveryBibitem{\clearfield{month}}
\AtEveryCitekey{\clearfield{month}}
\makeatletter
\DeclareDelimFormat[cbx@textcite]{nameyeardelim}{\addspace}
\makeatother
\renewcommand*{\finalnamedelim}{%
  %\ifnumgreater{\value{liststop}}{2}{\finalandcomma}{}% there really should be no funny Oxford comma business here
  \addspace\&\space}

%%% Change title format
\usepackage{color,titling,framed}
\usepackage[absolute,overlay]{textpos}
\setlength{\TPHorizModule}{1cm}
\setlength{\TPVertModule}{1cm}

\pretitle{%

\vspace*{-1.2cm}

\LARGE\bfseries}
\posttitle{\vspace*{0.3cm}\par}
\preauthor{\large}
\postauthor{\hfill}
\predate{\small}
\postdate{\vspace*{0.1cm}}

\raggedbottom

\usepackage[australian]{babel}
\date{Submission ID 217815520}



\begin{document}
\vspace*{-2cm}
\definecolor{shadecolor}{RGB}{210,210,210}
\begin{snugshade}\sffamily
\maketitle
\end{snugshade}\vspace*{0.5cm}
\definecolor{shadecolor}{RGB}{248,248,248}
\setstretch{1.2}


We thank the Editor and the three reviewers for their comments, which have helped substantially to improve the manuscript. In response, we have made considerable changes to the main manuscript as outlined below.

\section*{Major changes:}

Reviewers suggested reframing the problem space so there is more emphasis on the case study.

\begin{description}
\item[Reviewer 1:] \textcolor{violet}{How is this manuscript relevant for teachers and/or students in statistics and data science?}
\item[Reviewer 3:] \textcolor{violet}{The authors prominently cite (Chatfield 1985) for the definition of the term “IDA”. However, Chatfield fairly explicitly defines IDA to be a data summarizing and scrutinizing process; not a cleaning, scrubbing, or munging process. It seems curious to me that the whole paper is framed as being “IDA” based on the author’s unsupported statement that data cleaning should be considered party of IDA.}
\item[Reviewer 3:] \textcolor{violet}{My suggestion is that the authors reconfigure this paper to be presented as a pure case study. Ibelieve the work that has been done is valuable: the data cleaning process is often messy andunplanned, so a case study like this on a popular dataset would provide an excellent educational resource.}
\end{description}

Supporting information for the lack of data cleaning in papers.

\begin{description}
\item[Reviewer 3:] \textcolor{violet}{The authors also emphasize that “There are few research papers that document the data cleaning”. While I agree that data cleaning is under-emphasized and under-documented, it’s hardly true that there are not papers on the topic. A quick Google Scholar search for “data cleaning” unearths many scholarly papers and books on the principles of this process, none of which appear to be cited in this work.}
\end{description}

Outlier detection

\begin{description}
\item[Reviewer 3:] \textcolor{violet}{
Section 3.2.1 goes into detail on the authors’ approach to characterizing erroneous outliers in
the data. Their approach is reasonable, but it is not (as far as I can tell) based on any
established or tested procedure. This section once again presents a tension between the paper
as a case study versus a topical commentary: if it is a case study only, then the “common sense”
justification for the approach is appropriate, but if it is meant to establish future norms, these
modeling choices must be more formally supported. The reference to a “reasonable degree of
fluctuation” particularly struck me as a subjective or “ad hoc” claim. I am particularly a bit concerned by the statement on page 13: “The robust mixed model could
be the best model to be employed in this case. However, this method is too computationally
and memory expensive, especially for a large data set, like the NLSY79 data.” Surely, fitting a 
mixed-effect model to a dataset with only one predictor and 1188 individuals ought to be very
feasible?
}
\end{description}

We thank the reviewer for pointing out to us that the data contains information on the work experience -- this additional data is added in the extraction of the data in Figure 1. This information was compared with Singer \& Willet's original data. Comparison shown in Figure X shows that the information does not match up, however, there is a high correlation between the variables.

\section*{Minor changes:}

\begin{itemize}
\tightlist
\item
  We have corrected all minor grammatical errors pointed out by the reviewers.\\
\item
  We have made the descriptions of the steps more explicit.
\end{itemize}

\begin{description}
\item[Reviewer 1:]\textcolor{violet}{It's unclear from context what is being referred to from the Huebner et al 2020 citation.}
\end{description}

\begin{itemize}
\tightlist
\item
  We have fixed the sentences and the reference.
\end{itemize}

\textbackslash begin\{description\}

\item[Reviewer 1:]

\textbackslash textcolor\{violet\}\{What is dplyr used for? Are there tidyverse packages used but not mentioned? If not, splitting this sentence into two may make it clearer which package is used to what purpose. Also, I haven't tried it in code, but I imagine that the data could be tidied as described just using pivot\_longer, at least for the job number, year, and wage data. In other words, I don't see why dplyr or stringr would be needed for the data described on page 6.\}
\textbackslash end\{description\}

\begin{itemize}
\tightlist
\item
  We have split the sentences and described the use of each package mentioned.
\end{itemize}

\textbackslash begin\{description\}

\item[Reviewer 1:]

\textbackslash textcolor\{violet\}\{please clarify. \enquote{If either the hourly wage or hours worked is missing, we do not tally this.} I take \enquote{this} to mean \enquote{number\_of\_jobs}. But in row 2, total\_hours is missing, yet number\_of\_jobs is 1. The number of jobs (1) was tallied even though hours worked was missing.\}
\textbackslash end\{description\}

\begin{itemize}
\tightlist
\item
  We have clarified this that we only tally the number of jobs if the hourly wage is not missing.
\end{itemize}

\begin{description}
\item[Reviewer 2:] \textcolor{violet}{My primary comment is that many elements need to be more explicitly spelled out.}
\end{description}

\printbibliography

\end{document}
